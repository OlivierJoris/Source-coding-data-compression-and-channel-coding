\documentclass[a4paper, 11pt, oneside]{article}

\usepackage[utf8]{inputenc}
\usepackage[T1]{fontenc}
\usepackage[english]{babel}
\usepackage{array}
\usepackage{shortvrb}
\usepackage{listings}
\usepackage[fleqn]{amsmath}
\usepackage{amsfonts}
\usepackage{fullpage}
\usepackage{enumerate}
\usepackage{graphicx}
\usepackage{alltt}
\usepackage{indentfirst}
\usepackage{eurosym}
\usepackage{titlesec, blindtext, color}
\usepackage[table,xcdraw,dvipsnames]{xcolor}
\usepackage[unicode]{hyperref}
\usepackage{url}
\usepackage{float}
\usepackage{subcaption}
\usepackage[skip=1ex]{caption}

\definecolor{brightpink}{rgb}{1.0, 0.0, 0.5}

\usepackage{titling}
\renewcommand\maketitlehooka{\null\mbox{}\vfill}
\renewcommand\maketitlehookd{\vfill\null}

\newcommand{\ClassName}{ELEN-0060: Information and Coding Theory}
\newcommand{\ProjectName}{Project 2 - Source coding, data compression and \\ channel coding}
\newcommand{\AcademicYear}{2021 - 2022}

%%%% First page settings %%%%

\title{\ClassName\\\vspace*{0.8cm}\ProjectName\vspace{1cm}}
\author{Maxime Goffart \\180521 \and Olivier Joris\\182113}
\date{\vspace{1cm}Academic year \AcademicYear}

\begin{document}

%%% First page %%%
\begin{titlingpage}
{\let\newpage\relax\maketitle}
\end{titlingpage}

\thispagestyle{empty}
\newpage

%%%%%%%%%%%%%%%%%%%%%%%%%%%%%%%%%%%%%%%%%%

%%% Table of contents %%%
%\tableofcontents
%\newpage

%%%%%%%%%%%%%%%%%%%%%%%%%%%%%%%%%%%%%%%%%%

% CONTENT %



%%%%%%%%%%%%%%%%%%%%%%%%%%%%%%%%%%%%%%%%%%
\section{Channel coding}

\subsection{Question 16}
\paragraph{}In order to implement a function to read and display the given image, we used the methods \texttt{imread} and \texttt{imshow} provided by OpenCV.

%%%%%

\subsection{Question 17}
\paragraph{}To encode the image signal, we used a fixed-length binary code of 8 bits. We have chosen 8 bits because there are 256 (from 0 to 255) possible values, so we need $\lceil log_2(256) \rceil =8$.\\
The code is the binary representation of the grayscale value of each pixel.

%%%%%%%%%%%%%%%%%%%%%%%%%%%%%%%%%%%%%%%%%%

\end{document}